% Options for packages loaded elsewhere
\PassOptionsToPackage{unicode}{hyperref}
\PassOptionsToPackage{hyphens}{url}
%
\documentclass[
]{article}
\usepackage{lmodern}
\usepackage{amssymb,amsmath}
\usepackage{ifxetex,ifluatex}
\ifnum 0\ifxetex 1\fi\ifluatex 1\fi=0 % if pdftex
  \usepackage[T1]{fontenc}
  \usepackage[utf8]{inputenc}
  \usepackage{textcomp} % provide euro and other symbols
\else % if luatex or xetex
  \usepackage{unicode-math}
  \defaultfontfeatures{Scale=MatchLowercase}
  \defaultfontfeatures[\rmfamily]{Ligatures=TeX,Scale=1}
\fi
% Use upquote if available, for straight quotes in verbatim environments
\IfFileExists{upquote.sty}{\usepackage{upquote}}{}
\IfFileExists{microtype.sty}{% use microtype if available
  \usepackage[]{microtype}
  \UseMicrotypeSet[protrusion]{basicmath} % disable protrusion for tt fonts
}{}
\makeatletter
\@ifundefined{KOMAClassName}{% if non-KOMA class
  \IfFileExists{parskip.sty}{%
    \usepackage{parskip}
  }{% else
    \setlength{\parindent}{0pt}
    \setlength{\parskip}{6pt plus 2pt minus 1pt}}
}{% if KOMA class
  \KOMAoptions{parskip=half}}
\makeatother
\usepackage{xcolor}
\IfFileExists{xurl.sty}{\usepackage{xurl}}{} % add URL line breaks if available
\IfFileExists{bookmark.sty}{\usepackage{bookmark}}{\usepackage{hyperref}}
\hypersetup{
  hidelinks,
  pdfcreator={LaTeX via pandoc}}
\urlstyle{same} % disable monospaced font for URLs
\usepackage{color}
\usepackage{fancyvrb}
\newcommand{\VerbBar}{|}
\newcommand{\VERB}{\Verb[commandchars=\\\{\}]}
\DefineVerbatimEnvironment{Highlighting}{Verbatim}{commandchars=\\\{\}}
% Add ',fontsize=\small' for more characters per line
\newenvironment{Shaded}{}{}
\newcommand{\AlertTok}[1]{\textcolor[rgb]{1.00,0.00,0.00}{\textbf{#1}}}
\newcommand{\AnnotationTok}[1]{\textcolor[rgb]{0.38,0.63,0.69}{\textbf{\textit{#1}}}}
\newcommand{\AttributeTok}[1]{\textcolor[rgb]{0.49,0.56,0.16}{#1}}
\newcommand{\BaseNTok}[1]{\textcolor[rgb]{0.25,0.63,0.44}{#1}}
\newcommand{\BuiltInTok}[1]{#1}
\newcommand{\CharTok}[1]{\textcolor[rgb]{0.25,0.44,0.63}{#1}}
\newcommand{\CommentTok}[1]{\textcolor[rgb]{0.38,0.63,0.69}{\textit{#1}}}
\newcommand{\CommentVarTok}[1]{\textcolor[rgb]{0.38,0.63,0.69}{\textbf{\textit{#1}}}}
\newcommand{\ConstantTok}[1]{\textcolor[rgb]{0.53,0.00,0.00}{#1}}
\newcommand{\ControlFlowTok}[1]{\textcolor[rgb]{0.00,0.44,0.13}{\textbf{#1}}}
\newcommand{\DataTypeTok}[1]{\textcolor[rgb]{0.56,0.13,0.00}{#1}}
\newcommand{\DecValTok}[1]{\textcolor[rgb]{0.25,0.63,0.44}{#1}}
\newcommand{\DocumentationTok}[1]{\textcolor[rgb]{0.73,0.13,0.13}{\textit{#1}}}
\newcommand{\ErrorTok}[1]{\textcolor[rgb]{1.00,0.00,0.00}{\textbf{#1}}}
\newcommand{\ExtensionTok}[1]{#1}
\newcommand{\FloatTok}[1]{\textcolor[rgb]{0.25,0.63,0.44}{#1}}
\newcommand{\FunctionTok}[1]{\textcolor[rgb]{0.02,0.16,0.49}{#1}}
\newcommand{\ImportTok}[1]{#1}
\newcommand{\InformationTok}[1]{\textcolor[rgb]{0.38,0.63,0.69}{\textbf{\textit{#1}}}}
\newcommand{\KeywordTok}[1]{\textcolor[rgb]{0.00,0.44,0.13}{\textbf{#1}}}
\newcommand{\NormalTok}[1]{#1}
\newcommand{\OperatorTok}[1]{\textcolor[rgb]{0.40,0.40,0.40}{#1}}
\newcommand{\OtherTok}[1]{\textcolor[rgb]{0.00,0.44,0.13}{#1}}
\newcommand{\PreprocessorTok}[1]{\textcolor[rgb]{0.74,0.48,0.00}{#1}}
\newcommand{\RegionMarkerTok}[1]{#1}
\newcommand{\SpecialCharTok}[1]{\textcolor[rgb]{0.25,0.44,0.63}{#1}}
\newcommand{\SpecialStringTok}[1]{\textcolor[rgb]{0.73,0.40,0.53}{#1}}
\newcommand{\StringTok}[1]{\textcolor[rgb]{0.25,0.44,0.63}{#1}}
\newcommand{\VariableTok}[1]{\textcolor[rgb]{0.10,0.09,0.49}{#1}}
\newcommand{\VerbatimStringTok}[1]{\textcolor[rgb]{0.25,0.44,0.63}{#1}}
\newcommand{\WarningTok}[1]{\textcolor[rgb]{0.38,0.63,0.69}{\textbf{\textit{#1}}}}
\setlength{\emergencystretch}{3em} % prevent overfull lines
\providecommand{\tightlist}{%
  \setlength{\itemsep}{0pt}\setlength{\parskip}{0pt}}
\setcounter{secnumdepth}{-\maxdimen} % remove section numbering
\ifluatex
  \usepackage{selnolig}  % disable illegal ligatures
\fi

\author{}
\date{}

\begin{document}

\hypertarget{maxwells-demon}{%
\section{Maxwell's Demon}\label{maxwells-demon}}

\hypertarget{neil-d-lawrence}{%
\subsubsection{Neil D. Lawrence}\label{neil-d-lawrence}}

\hypertarget{2020-01-01}{%
\subsubsection{2020-01-01}\label{2020-01-01}}

Our Maxwell's demon is formed by a chamber with a dividing membrane. The
chamber contains particles, all of equall mass \(m=1\). The chamber is
split in two parts by a membrane. The membrane's effect on the particles
is controlled by the demon.

The velocities of the particles are given by, \(\mathbf{v} = \{ v_i\}\).
Each particle sits on either the right or left hand side of the
membrane. The position of each particle is denoted by
\(\mathbf{s} = \{ s_i \}\), with \(s_i=0\) if the \(i\)th particle is on
the left hand side of the membrane, and \(s_i = 1\) if the \(i\)th
particle is on the right hand side of the membrane.

The demon has the power to allow particles to pass through the membrane.
Any particle which is allowed to pass through the membrane is labelled
with \(d_i = 1\). Particles that don't pass the membrane rebound with an
elastic collision. The full set of state variables associated with the
demon's decisions are given by \(\mathbf{d} = \{ d_i \}\).

We are interested in the joint distribution of these states, \[
p(\mathbf{v}, \mathbf{s}, \mathbf{d}).
\]

We assume that the particles are colliding with elastic collisions, so
that the kinetic energy of the system is conserved and we have, \[
E = \frac{1}{2}\sum_{i}v_i^2.
\] We are intererested in the stationary distribution of this system.
That stationary distribution will occur when equal number of particles
pass from the left hand side of the membrane to the right hand side and
when particles passing the membrane loose the demon mark. These
condition are expressed by \[\begin{align*}
-\eta\sum_{i} (1-s_i)(1-d_i) v_i & = -\eta\sum_{i} s_i d_i v_i,\\
\eta\sum_{i} (1-s_i) d_i v_i & = \eta\sum s_i(1- d_i) v_i,\\
\eta\sum_{i} (1-s_i) d_i v_i & = -\eta\sum_{i} s_i d_i v_i.
\end{align*}\]

Finally, we have the demon, who selects which particles will be able to
pass through the membrane. The demon operates on the particles according
to the side of the membrane they are on and the speed they are
travelling. We introduce an indicator function, \(I(\cdot)\), that
depends on the side, \(s_i\), and the velocity, \(v_i\), which returns a
value of zero or one to set \(d_i\), \[
d_i = I(v_i, s_i).
\] We also set \(I(v_i, s_i=0) = 1- I(v_i, s_i=1)\), so that the value
of \(d_i\) is reversed on the two different sides of the membrane.

These constraints allow us to build the stationary distribution through
the principle of maximum entropy. The idea of maximum entropy is that
the distribution will tend to its maximum entropy variant over time, so
we have, \[\begin{align*}
p(\mathbf{v}, \mathbf{s}, \mathbf{d}) \propto & \exp\left(\frac{\lambda_1}{2}\sum_i v_i^2 \right. \\ & 
+ \lambda_2\left(\sum_{i} (1-s_i)(1-d_i) v_i - \sum_{i} s_i d_i v_i\right) \\ &
+ \lambda_3\left(\sum_{i} s_i(1-d_i)v_i - \sum_{i} (1-s_i) d_i v_i\right) \\ &
+ \lambda_4\left(\sum_{i} s_i d_i v_i + \sum_{i} (1-s_i)d_i v_i\right) \\ &
+ \left. \sum_i \mu_i \left(d_i - I(v_i, s_i)\right)\right),
\end{align*}\] where \(\lambda_1\), \(\lambda_2\) and
\(\left\{\mu_i\right\}\) are the Lagrange multipliers associated with
the conditions on the moments.

The distribution factorises over particles so we can consider the
distribution for an individual particle which has the form,
\[\begin{align*}
p(v_i, s_i, d_i) \propto & \exp\left(\frac{\lambda_1}{2} v_i^2 \right. \\ 
& + \lambda_2 \left((1-s_i)(1-d_i) -s_i d_i\right)v_i \\
& + \lambda_3 \left(s_i(1-d_i) - (1-s_i) d_i\right)v_i \\
& + \lambda_4\left( s_i d_i + (1-s_i)d_i \right)v_i \\
& + \left. \mu_i \left(d_i - I(v_i, s_i)\right)\right)
\end{align*}\]

We consider the distribution for the different values of \(s_i\) and
\(d_i\). \[
\begin{align*}
p(v_i | s_i=0, d_i=0) & \propto \exp\left(\frac{\lambda_1}{2} v_i^2 + \lambda_2 v_i\right) \exp\left(-\mu_i I(v_i, s_i=0)\right) \\
p(v_i | s_i=1, d_i=0) & \propto \exp\left(\frac{\lambda_1}{2} v_i^2 + \lambda_3 v_i\right) \exp\left(-\mu_i I(v_i, s_i=1)\right) \\
p(v_i | s_i=0, d_i=1) & \propto \exp\left(\frac{\lambda_1}{2} v_i^2 + (\lambda_4 - \lambda_3) d_i v_i\right)\exp\left(\mu_i\left(1-I(v_i, s_i=0)\right)\right) \\
p(v_i | s_i=1, d_i=1) & \propto \exp\left(\frac{\lambda_1}{2} v_i^2 + (\lambda_4 - \lambda_2) d_i v_i\right)\exp\left(\mu_i\left(1-I(v_i, s_i=1)\right)\right)
\end{align*}
\]

\[\begin{align*}
f(v_i)\exp\left( k I(v_i, s_i=0) \right) & = I(v_i, s_i=0)\exp(k)f(v_i) +  (1-I(v_i, s_i=0)) f(v_i) \\
& = \exp(k)I(v_i, s_i=0)f(v_i) + I(v_i, s_i=1) f(v_i).
\end{align*}\]

\[
\begin{align*}
p(v_i | s_i=0, d_i=0) & \propto \exp\left(-\mu_i\right) I(v_i, s_i=0)\exp\left(\frac{\lambda_1}{2} v_i^2\right) + I(v_i, s_i=1)\exp\left(\frac{\lambda_1}{2} v_i^2\right) \\
p(v_i | s_i=1, d_i=0) & \propto I(v_i, s_i=0)\exp\left(\frac{\lambda_1}{2} v_i^2\right) + \exp\left(-\mu_i\right) I(v_i, s_i=1)\exp\left(\frac{\lambda_1}{2} v_i^2\right)  \\
p(v_i | s_i=0, d_i=1) & \propto I(v_i, s_i=0)\exp\left(\frac{\lambda_1}{2} v_i^2 + \lambda_2 d_i v_i\right) + \exp(\mu_i)I(v_i, s_i=1)\exp\left(\frac{\lambda_1}{2} v_i^2 + \lambda_2 d_i v_i\right) \\
p(v_i | s_i=1, d_i=1) & \propto \exp(\mu_i)I(v_i, s_i=0)\exp\left(\frac{\lambda_1}{2} v_i^2 + \lambda_2 d_i v_i\right) + I(v_i, s_i=1)\exp\left(\frac{\lambda_1}{2} v_i^2 + \lambda_2 d_i v_i\right)
\end{align*}
\]

Which we can rewrite as

\[\begin{align*}
p(v_i, s_i, d_i) \propto & \exp\left( s_i (1-d_i) \frac{\lambda_1}{2}v_i^2\right) \\
& \times \exp\left((1-s_i)d_i \frac{\lambda_1}{2} v_i^2\right) \\
& \times \exp\left(s_i d_i \left(\frac{\lambda_1}{2} v_i^2 + \lambda_2 v_i\right)\right) \\
& \times \exp\left((1-s_i)(1-d_i) \left(\frac{\lambda_1}{2} v_i^2- v_i\right)\right)
\end{align*}\]

where the components in the product are switched in according to the
configuration of the binary variables \(d_i\) and \(s_i\). Each
component can be recognised as a Gaussian,

\(\newcommand{\gaussianDist}[3]{\mathcal{N}\left(#1|#2, #3\right)}\)
\[\begin{align*}
p(v_i, s_i, d_i) = & \frac{1}{4}\gaussianDist{v_i}{0}{-\lambda_1^{-1}}^{s_i (1-d_i)} \\
& \times \gaussianDist{v_i}{0}{-\lambda_1^{-1}}^{(1 -s_i) d_i} \\
& \times \gaussianDist{v_i}{\frac{\lambda_2}{2\lambda_1}}{-\lambda_1^{-1}}^{s_i d_i}\\
& \times \gaussianDist{v_i}{-\frac{\lambda_2}{2\lambda_1}}{-\lambda_1^{-1}}^{(1-s_i) (1-d_i)}.
\end{align*}\]

And the marginal distribution over velocities is therefore a
\emph{mixture of Gaussians}. \[
\begin{align*}
p(v_i) = & \frac{1}{2}\gaussianDist{v_i}{0}{-\lambda_1^{-1}} 
 + \frac{1}{4}\gaussianDist{v_i}{\frac{\lambda_2}{2\lambda_1}}{-\lambda_1^{-1}}
 + \frac{1}{4}\gaussianDist{v_i}{-\frac{\lambda_2}{2\lambda_1}}{-\lambda_1^{-1}}.
\end{align*}
\] We are interested in the \emph{entropy} of this distribution. The
entropy of a mixture can be problematic, we first decompose it into the
entropy of the components and a mutual information term. \[
H(v_i) =  - \int \sum_m p(v_i | m) p(m) \log \sum p(v_i | m) p(m) \text{d}v_i
\] \[
H(v_i) =  - \int \sum_m p(v_i | m) p(m) \log p(v_i | m)  \text{d}v_i - \int \sum_m p(v_i | m) p(m) \log \frac{\sum_m p(v_i | m) p(m)}{p(v_i | m)}  \text{d}v_i
\] \[
H(v_i) =  - \sum_m  p(m)\int p(v_i | m) \log p(v_i | m)  \text{d}v_i - \int \sum_m p(v_i , m) \log \frac{p(v_i) p(m)}{p(v_i,m)}  \text{d}v_i
\] \[
H(v_i) =  - \sum_m  p(m)\int p(v_i | m) \log p(v_i | m)  \text{d}v_i + \int \sum_m p(v_i , m) \log \frac{p(v_i,m)}{p(v_i) p(m)}  \text{d}v_i
\]

Consider some scenarios.

\begin{enumerate}
\item
  Let's assume that no particles are allowed to pass the membrane, this
  means \(d_i=0\). In this case the system separates into two
  compartments, with mean zero and variance \(-\lambda_1^2\) for each
  compartment.
\item
  Let's assume that all particles are always allowed to pass the
  membrane. Now \(d_i=1\) for all \(i\).
\end{enumerate}

and distribution no longer depends on \(s_i\)Now if we imagine that the
demon is capable of observing which particles are on which side and
which

\begin{Shaded}
\begin{Highlighting}[]

\end{Highlighting}
\end{Shaded}


\end{document}
