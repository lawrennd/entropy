% Options for packages loaded elsewhere
\PassOptionsToPackage{unicode}{hyperref}
\PassOptionsToPackage{hyphens}{url}
%
\documentclass[twocolumn]{article}
\usepackage[center]{titlesec}
\renewcommand{\thesection}{\Roman{section}}
\titleformat{\section}
  {\large\centering}
  {\thesection}{1em}{}
\usepackage{lmodern}
\usepackage{amssymb,amsmath}
\usepackage{times}
\usepackage{ifxetex,ifluatex}
\ifnum 0\ifxetex 1\fi\ifluatex 1\fi=0 % if pdftex
  \usepackage[T1]{fontenc}
  \usepackage[utf8]{inputenc}
  \usepackage{textcomp} % provide euro and other symbols
\else % if luatex or xetex
  \usepackage{unicode-math}
  \defaultfontfeatures{Scale=MatchLowercase}
  \defaultfontfeatures[\rmfamily]{Ligatures=TeX,Scale=1}
\fi
% Use upquote if available, for straight quotes in verbatim environments
\IfFileExists{upquote.sty}{\usepackage{upquote}}{}
\IfFileExists{microtype.sty}{% use microtype if available
  \usepackage[]{microtype}
  \UseMicrotypeSet[protrusion]{basicmath} % disable protrusion for tt fonts
}{}
\makeatletter
\@ifundefined{KOMAClassName}{% if non-KOMA class
  \IfFileExists{parskip.sty}{%
    \usepackage{parskip}
  }{% else
    \setlength{\parindent}{0pt}
    \setlength{\parskip}{6pt plus 2pt minus 1pt}}
}{% if KOMA class
  \KOMAoptions{parskip=half}}
\makeatother
\usepackage{xcolor}
\IfFileExists{xurl.sty}{\usepackage{xurl}}{} % add URL line breaks if available
\IfFileExists{bookmark.sty}{\usepackage{bookmark}}{\usepackage{hyperref}}
\hypersetup{
  pdftitle={Gibbs vs Boltzmann Entropies},
  pdfauthor={E. T. Jaynes},
  hidelinks,
  pdfcreator={LaTeX via pandoc}}
\urlstyle{same} % disable monospaced font for URLs
\setlength{\emergencystretch}{3em} % prevent overfull lines
\providecommand{\tightlist}{%
  \setlength{\itemsep}{0pt}\setlength{\parskip}{0pt}}
\ifluatex
  \usepackage{selnolig}  % disable illegal ligatures
\fi

\title{Gibbs vs Boltzmann Entropies\footnote{Supported by the National
  Science Foundation Grant XSF G23778.}}
\author{E. T. J\textsc{aynes}}
\date{}
\address{Department, of Physics\emph{,} Washington University, St. Louis\emph{,}
Missouri}
\note{(Received 27 March 1964; in final form, 5 November 1964)}

\begin{document}
\maketitle


\begin{abstract}
The status of the Gibbs and Boltzmann expressions for entropy has been
a matter of some confusion hi the literature. We show that; (1) the
Gibbs 77 function yields the correct entropy as defined in
phenomenological thermodynamics ; (2) the Boltzmann $H$ yields an
"entropy'' that is in error by a nonnegligible amount whenever
interparticle forces affect thermodynamic properties; (3) Boltzmann's
other interpretation of entropy $S = k\log W$, is consistent with
the Gibbs $H$, and derivable from it; (4) the Boltzmann $H$
theorem does not constitute a demonstration of the second law for dilute
gases; (5) the dynamical invariance of the Gibbs $H$ gives a simple
proof of the second law for arbitrary inter particle forces; (6) the
second law is a special case of a general requirement for any
macroscopic process to be experimentally reproducible. Finally, the
``anthropomorphic'' nature of entropy, on both the statistical and
phenomenological levels, is stressed.
\end{abstract}

\section{Introduction}\label{sec-introduction}

In the writer's 1962 Brandeis lectures\footnote{\emph{Statistical
  Physics} (1962 Brandeis Theoretical Physics Lectures, Vol. 3), edited
  by K. W. Ford (W. A. Benjamin, Inc., New York, 1963), Chap. 4. Note
  that typographical errors occur in Eqs. 20, 49, 74, 78, 94, and the
  inequality preceding Eq. 90.} on statistical mechanics, the Gibbs and
Boltzmann expressions for entropy were compared briefly, and it was
staled that the Gibbs formula gives the correct entropy, as defined in
phenomenological thermodynamics, while the Boltzmann $H$ expression
is correct only in the case of an ideal gas. However, there is a school
of thought which holds that the Boltzmann expression is directly related
to the entropy, and the Gibbs' one simply erroneous. This belief can be
traced back to the famous Ehrenfest review article,\footnote{P.
  Ehrenfest and T. Ehrenfest, Encykl. Math. Wiss., IV 2, II, Issue 6
  (1912) Reprinted in \emph{Paul Ehrenfest}, \emph{Collected Scientific
  Papers}, edited by M. J. Klein (Xorth- Holiand Press, Amsterdam,
  1959). English translation by M. J. Moravcsik, \emph{The Conceptual
  Foundations of the Statistical Approach in Mechanics} (Cornell
  University Press, Ithaca, New York, 1959).} which severely criticized
Gibbs' methods.

While it takes very little thought to see that objections to the Gibbs
$H$ are immediately refuted by the fact that the Gibbs canonical
ensemble does yield correct thermodynamic predictions, discussion with a
number of physicists has disclosed a more subtle, but. more widespread,
misconception. The basic inequality of the Gibbs and Boltzmann $H$
functions, to be derived in Sec. \ref{sec-the-basic-inequality}, was accepted as mathematically
correct; but it was thought that, in consequence of the ``laws of large
numbers'' the difference between them would be practically negligible in
the limit of large systems.

Now it is true that there are many different entropy expressions that go
into substantially the same thing in this limit; several examples were
given by Gibbs. However, the Boltzmann expression is not one of them; as
we prove in Sec. \ref{sec-canonical-ensemble}, the difference is a direct measure of the effect
of interparticle forces on the potential energy and pressure, and
increases proportionally' to the size of the system.

Failure to recognize the fundamental role of the Gibbs $H$ function
is closely related to a much deeper confusion about entropy,
probability, and irreversibility in general. For example, the Boltzmann
$H$ theorem is almost universally equated to a demonstration of the
second law of thermodynamics for dilute gases, while ever since the
Ehrenfest criticisms, it has been claimed repeatedly that the Gibbs
$H$ cannot be related to the entropy because it is constant in
time.

Closer inspection reveals that the situation is very different. Merely
to exhibit a mathematical quantity which tends to increase is not
relevant to the second law unless one demonstrates that this quantity is
related to the entropy as measured experimentally. But neither the Gibbs
nor the Boltzmann $H$ is so related for any distribution other than
the equilibrium (i.e., canonical) one. Consequently, although
Boltzmann's $H$ theorem does show the tendency of a gas to go into
a Maxwellian velocity distribution, this is not the same thing as the
second law, which is a statement of experimental fact about the
direction in which the observed \emph{macroscopic} quantities
\((P,V,T)\) change.

Past attempts to demonstrate the second law for systems other than
dilute gases have generally tried to retain the basic idea of the
Boltzmann $H$ theorem. Since the Gibbs $H$ is dynamically
constant, one has resorted to some kind of coarse-graining operation,
resulting in a new quantity ~~\(\overline{H}\)\emph{,} which tends to
decrease. Such attempts cannot achieve their purpose, because (a)
mathematically, the decrease in ~\(\overline{H}\) is due only to the
artificial coarse-graining operation and it cannot, therefore have any
physical significance; (b) as in the Boltzmann $H$ theorem, the
quantity whose increase is demonstrated is not the same thing as the
entropy. For the fine-grained and coarse-grained probability
distributions lead to just the same predictions for the observed
macroscopic quantities, which alone determine the experimental entropy;
the difference between \(H\) and \(\overline{H}\) is characteristic, not
of the macroscopic state, but of the particular way in which we choose
to coarse-grain. Any really satisfactory demonstration of the second law
must therefore be based on a different approach than coarse-graining.

Actually, a demonstration of the second law, in the rather specialized
situation visualized in the aforementioned attempts, is much simpler
than any $H$ theorem. Once we accept the well-established
proposition that the Gibbs canonical ensemble does yield the correct
equilibrium thermodynamics, then there is logically no room for any
assumption about which quantity represents entropy; it is a question of
mathematically demonstrable fact. But as soon as we have understood the
relation between Gibbs' $H$ and the experimental entropy, Eq. (\ref{eqn-seventeen})
below, it is immediately obvious that the constancy of Gibbs' \emph{H,}
far from creating difficulties, is precisely the dynamical property we
need for the proof.

It is interesting that, although this field has long been regarded as
one of the most puzzling and controversial parts of physics, the
difficulties have not been mathematical. Each of the above assertions is
proved below or in the Brandeis lectures, using only a few lines of
elementary mathematics, all of which was given by Gibbs. It is the
enormous \emph{conceptual} difficulty of this field which has retarded
progress for so long. Readers not familiar with recent developments may,
I hope, be pleasantly surprised to see how clear and basically simple
these problems have now become, in several respects. However, as we will
see, there are still many complications and unsolved problems.

Inspection of several statistical mechanics textbooks showed that, while
most state the formal relations correctly, their full implications are
never noted. Indeed, while all textbooks give extensive discussions of
Boltzmann's \(H\), some recent ones fail to mention even the existence
of the Gibbs \(H\)\emph{.}\footnote{A notable exception is the
  monumental work of R. C. Tolman, \emph{The Principles of Statistical
  Mechanics} (Oxford University Press, London, 1938). Tolman repeatedly
  stresses the superiority of Gibbs' approach, although he still
  attempts to base the second law on coarse-graining.} I was unable to
find any explicit mathematical demonstration of their difference. It
appeared, therefore, that the following note might be pedagogically
useful.

\section{The Basic Inequality}\label{sec-the-basic-inequality}

We consider, as usual, a monoatomic fluid of \(N\) particles. The
ensemble is defined by the \(N\)\emph{-}particle distribution function,
or Liouville function,
\(W_{N}(x_{1},p_{1};x_{2},p_{2};\ldots;x_{N},p_{N};t)\) which gives the
probability density in the full phase space of the system. The Gibbs
$H$ is then

\begin{equation}
H_{G} = {\int_{}^{}{W_{N}\log W_{N}}} d\tau
\label{eqn-one}
\end{equation}

and the corresponding Boltzmann $H$ is

\begin{equation}
H_{B} = N\int w_{1}\text{log\ }w_{1}d\tau_{1},
\label{eqn-two}
\end{equation}

\begin{equation}
w_{1}\left( x_{1},p_{1};t \right) = \int W_{N}d\tau_{- 1}
\label{eqn-three}
\end{equation}

Here and in the following, we use the notation:
\(d\tau \equiv d^{3}x_{1}\ldots d^{3}p_{N}\),\(d\tau_{1} \equiv d^{3}x_{1}d^{3}p_{1}\),\(d\tau_{- 1} \equiv d^{3}x_{2}\ldots d^{3}p_{N}\)
to stand for phase-volume elements in the full phase space, the space of
one particle, and the space of all particles except one, respectively.

Both the Gibbs and Boltzmann $H$ functions are often defined in
slightly different ways, in which one uses distribution functions with
different normalizations. This changes the numerical values by additive
constants which, for fixed $N$, are independent of the
thermodynamic state and therefore not relevant to the present
discussion. These additive constants are important, however, in
connection with the ``Gibbs paradox'' about entropy of mixing, and the
resolution of this paradox by quantum statistics is well known. The
distribution functions used above are understood to be \emph{probability
densities}; i.e., normalized according to
\(\int W_{N}d\tau = \int w_{1}d\tau_{1} = 1\).

Using (\ref{eqn-three}) and the fact that $W_N$ is symmetric under permutations of particle labels, we can write $H_B$ in a more symmetrical form
%
\begin{align*}
H_{B} & = N\ \int W_{N}\log{w_{1}\left( x_{1},p_{1} \right){d\tau}} \\
& = \int W_{N}\log{\left\lbrack w_{1}\left( 1 \right)\ldots w_{1}\left( N \right) \right\rbrack{d\tau}},
\end{align*}
%
where we use the abbreviation: $(i) \equiv (x_i, p_i)$. We have, then,
%
\begin{equation}
H_{B} - H_{G} = \int W_{N}\log \left\lbrack \frac{w_{1}\left( 1 \right)\ldots w_{1}\left( N \right)}{W_{N}\left( 1\ldots N \right)} \right\rbrack d\tau. 
\label{eqn-four}
\end{equation}

Now on the positive real axis, $\log x \leq (x-1)$, with equality if and only if $x=1$. Therefore
\[
H_{B} - H_{G} \leq \int W_{N}\left\lbrack \frac{w_{1}\left( 1 \right)\ldots w_{1}\left( N \right)}{W_{N}\left( 1\ldots N \right)} - 1 \right\rbrack d\tau = 0,
\]
and we have proved

\emph{Theorem 1}: The Gibbs and Boltzmann $H$ functions satisfy the inequality
\begin{equation}
H_{B} \leq H_{G},
\label{eqn-five}
\end{equation}
with equality if and only if $W_N$ factors ``almost everywhere'' into a product of single-particle functions
\[
W_{N}\left( 1\ldots N \right) = w_{1}\left( 1 \right)\ldots w_{1}\left( N \right).
\]

\section{Canonical Ensemble}\label{sec-canonical-ensemble}

Theorem 1 holds for any symmetrical $W_N$. The magnitude of the difference ($H_G - H_B)$ depends on the distribution function, and we are particularly interested in the case of thermal equilibrium, represented by the canonical distribution $W_N \sim \exp (-\beta H)$, where $\beta = (kT)^{-1}$ and $H$ is the Hamiltonian, taken of the form
%
\begin{equation}
H = \sum_{i = 1}^{N}\frac{p_{i}^{2}}{2m} + V\left( x_{1}\ldots x_{N} \right),
\label{eqn-six}
\end{equation}
%
where the potential-energy function \(V(x_{1}\ldots x_{N})\) is a
symmetrical function of the particle coordinates, which we suppose for
simplicity depends only the relative coordinates (relaxing this
restriction by adding gravitational potential energy leads to a number
of interesting results, but does not change the conclusions of this
section). More explicitly, we have
%
\begin{align}
W_{N} = & \left( \frac{\beta}{2\pi m} \right)^{3N/2}Q^{- 1} \nonumber \\
& \times \exp{\left\lbrack - \beta V\left( x_{1}\ldots x_{N} \right) - \beta\sum_{i}^{}\frac{p_{i}^{2}}{2m} \right\rbrack,}
\label{eqn-seven}
\end{align}
%
where
%
\begin{align}
Q\left( \beta,\Omega \right) & \equiv \int_{\Omega}^{}{\exp{\left( - \beta V \right)d^{3}x_{1}\ldots d^{3}x_{N}}} \nonumber \\ & = \Omega\int_{\Omega}^{}{\exp{\left( - \beta V \right)d^{3}x_{2}\ldots d^{3}x_{N}}}
\label{eqn-eight}
\end{align}
%
is the ``configuration integral,'' and in the last expression we have
made use of the fact that $V$ depends only on relative coordinates,
and supposed the range of interparticle forces negligibly small compared
to the size of the container, so that the final integration supplies
only a factor \(\Omega\). From (\ref{eqn-three}), the corresponding single-particle function is then
%
\begin{equation}
w_{1}\left( x,p \right) = {(\beta/2\pi m)}^{3/2}\Omega^{- 1}\exp{( - \beta p^{2}/2m).}
\label{eqn-nine}
\end{equation}

We therefore have
%
\[
\left\lbrack w_{1}\left( 1 \right)\ldots w_{1}\left( N \right) \right\rbrack/\ \ W_{N}(1\ldots N) = Q\Omega^{- N}e^{\beta N},
\]
%
and (\ref{eqn-four}) reduces to
%
\begin{equation}
H_{B} - H_{G} = \log Q - N\log\Omega + \beta\left\langle V \right\rangle,
\label{eqn-ten}
\end{equation}
%
where the angular brackets \(\left\langle \right\rangle\) denote the
canonical ensemble average. It is also true that
%
\begin{align}
\left\langle V \right\rangle & = - \partial\log Q/\partial\beta, \nonumber \\
\beta\left\langle P \right\rangle & = \partial\log Q/\partial\Omega,
\label{eqn-eleven}
\end{align}
%
where \(P\) is the pressure; Eq. (11) are well-known identities of the
canonical ensemble. From (10), (11), we thus find that on an
infinitesimal change of state,
%
\begin{equation}
d\left( H_{B} - H_{G} \right) = \beta d\left\langle V \right\rangle + \beta\left\lbrack \left\langle P \right\rangle - P_{0} \right\rbrack d\Omega
\label{eqn-twelve}
\end{equation}
%
where \(P_{0} \equiv NkT/\Omega\) is the pressure of an ideal gas with
the same temperature and density. Introducing the ``entropies''
\(S_{i} = - kH_{i}\) and integrating (\ref{eqn-twelve}) over a reversible path (i.e.,
a locus of equilibrium states), we see that the difference varies
according to
%
\begin{align}
\left( S_{G} - S_{B} \right)_{2} - \left( S_{G} \right. & \left. - S_{B} \right)_{1}  \nonumber \\
 & = \int_{1}^{2}\frac{d\left\langle V \right\rangle + \left\lbrack \left\langle P \right\rangle - P_{0} \right\rbrack d\Omega}{T}.
\label{eqn-thirteen}
\end{align}
%
Now from (\ref{eqn-nine}) using \(\left\langle p^{2} \right\rangle = 3mkT\), we find that
%
\[
S_{B} = \frac{3}{2}\text{Nk}\log\left( 2\pi mkT \right) + Nk\log\Omega + \frac{3}{2}Nk,
\]
%
from which
%
\[
\left( \frac{\partial S_{B}}{\partial T} \right)_{\Omega}dT = \frac{3}{2}\frac{\text{NkdT}}{T} = \frac{d\left\langle K \right\rangle}{T},\ 
\]
where \(\left\langle K \right\rangle = \frac{3}{2}\text{NkT}\) is the
total kinetic energy. Over the reversible path (\ref{eqn-thirteen}) the Boltzmann entropy therefore varies according to
%
\begin{equation}
\left( S_{B} \right)_{1} - \left( S_{B} \right)_{2} = \int_{1}^{2}\frac{d\left\langle K \right\rangle + P_{0}d\Omega}{T},
\label{eqn-fourteen}
\end{equation}
%
and from (\ref{eqn-thirteen}), (\ref{eqn-fourteen}) we finally have for the Gibbs entropy
%
\begin{align}
\left( S_{G} \right)_{2} - \left( S_{G} \right)_{1} & = \int_{1}^{2}\frac{d\left\langle K + V \right\rangle + \left\langle P \right\rangle d\Omega}{T} \nonumber \\& = \ \int_{1}^{2}{\frac{\text{dQ}}{T}.}
\label{eqn-fifteen}
\end{align}

Equations (\ref{eqn-fourteen}), (\ref{eqn-fifteen}) are the main
results sought. From them it is clear that (a) the ``Boltzmann
entropy'' is the entropy of a fluid with the same density and
temperature, but without interparticle forces; it completely neglects
both the potential energy and the effect of interparticle forces on
the pressure; (b) the Gibbs entropy is the correct entropy as defined
in phenomenological thermodynamics, which takes into account all the
energy and the total pressure, and is therefore equally valid for the
gas or condensed phases; (c) the difference between them is \emph{not}
negligible for any system in which interparticle forces have any
observable effect on the thermodynamic properties. If the system
exhibits an equation of state or heat capacity different from those of
an ideal gas, the Boltzmann entropy will be in error by a
corresponding amount.

\section{The Second Law}\label{sec-the-second-law}

We can now demonstrate the second law very easily, for the specialized
case usually considered. The following argument can be greatly
generalized, although we do not do so here.

It is well known\textsuperscript{1} that the canonical distribution (\ref{eqn-seven})
is uniquely determined by a variational property; over all distributions
\(W_{N}\) that agree with the experimental energy \(U\), in the sense
that the mean value of the Hamiltonian is
%
\begin{equation}
\left\langle H \right\rangle \equiv \int W_{N}H{d\tau} = U,
\label{eqn-sixteen}
\end{equation}
%
the Gibbs \(H\) attains an absolute minimum for the canonical
distribution. For this case, we have just shown that, if the arbitrary
additive constant is properly adjusted at a single point, then the Gibbs
entropy \(S_{G} = - kH_{G}\) will be the same as the experimental
entropy at all points. Therefore, the general relation between \(S_{G}\)
and the experimental entropy \(S_{e}\) is: over all distributions
\(W_{N}\) that agree with the experimental energy in the sense of (\ref{eqn-sixteen}),
we have
%
\begin{equation}
S_{G} \leq S_{e}
\label{eqn-seventeen}
\end{equation}
%
with equality if, and only if, \(S_{G}\) is computed from the canonical
distribution (\ref{eqn-seven}).

At time \(t = 0\), let our system be in complete thermal equilibrium so
that all its reproducible macroscopic properties are represented by the
canonical distribution; then the equality holds in (\ref{eqn-seventeen}). Now force the
system to carry out an adiabatic change of state (i.e., one involving no
heat flow to or from its environment), by applying some time-dependent
term in the Hamiltonian (such as moving a piston or varying a magnetic
field). It is well known that the \emph{N-}particle distribution
function varies according to the Liouville equation
~\(\dot{W_{N}} = \left\{ H\left( t \right),W_{N} \right\}\) where the
right-hand side is the Poisson bracket; and in consequence \(H_{G}\)
remains constant.

At a later time \(t^{\prime}\)\emph{,} the system is allowed to come once
more, but still adiabatically, to equilibrium (which means
experimentally that macroscopic quantities such as pressure or
magnetization are no longer varying), so that a new experimental entropy
\(S_{e}^{\prime}\) can be defined. If the time-developed distribution
function \(W_{N}(t^{\prime})\) leads to a correct prediction of the new
energy \(U^{\prime}\) in the sense of (\ref{eqn-sixteen}), then the inequality (\ref{eqn-seventeen}) still
holds. The fact that \(H_{G}\) is a constant of the motion then gives
\(S_{e} \leq S_{e}^{\prime}\), which is the second law.

\section{Intuitive Meaning of the Second Law}\label{sec-intiutive-meaning-of-the-second-law}

The above proof has the merit of being almost unbelievably short, but
partly for that reason, the physical basis of the second law is not made
clear. In the following we are not trying to give a rigorous
mathematical demonstration; that has just been done. We are trying
rather to exhibit the basic intuitive reason for the second law. We
recall Boltzmann's original conception of entropy as measuring the
logarithm of phase volume associated with a macroscopic state. If
Boltzmann's interpretation \(S = k\log\text{W\ }\)is to be compatible
with Gibbs' \(S = - k\ H_{G}\) it must be true that the quantity
\(W \equiv \exp{( - H_{G}})\) measures, in some sense, the phase volume
of ``reasonably probable'' microstates.

Such a connection can be established as follows. Define a
``high-probability'' region \(R\) of phase space, consisting of all
points where \(W_{N} > C\), and choose the constant \(C\) so that the
total probability of finding the system somewhere in this region is
\((1 - \epsilon)\), where \(0 < \epsilon < 1\). Call the phase volume of
this region \(W(\epsilon)\); in equations,

\[
\int_{R}W_{N} d\tau = 1 - \epsilon,
\]

\[
\int_{R} d\tau = W(\epsilon)
\]

Evidently, with a continuously varying probability density \(W_{N}\) it
is not strictly meaningful to speak of the ``phase volume of an
ensemble,'' without qualifications; but the ``minimum phase volume of
50\% probability'' or the ``minimum phase volume of 99\% probability''
do have precise meanings.

A remarkable limit theorem first noted by Shannon\footnote{E. T. Jaynes,
  Phys. Rev. 108, 171 (1957).} \footnote{C. E. Shannon, Bell Syst. Tech.
  J. 27, 379, 623 (1948); reprinted in C. E. Shannon and W. Weaver,
  \emph{The Mathematical Theory of Communication} (University of
  Illinois Press, Urbana, Illinois, 1949). See, particularly, Sec. 21.}
shows that for most purposes the particular probability level e is
unimportant. We quote the result without proof; it is an adaptation of
the fundamental ``asymptotic equipartition property'' (AEP) of
Information Theory.\footnote{A. Feinstein, \emph{Foundations of
  Information Theory} Chap. 6. (McGraw-Hill Book Company, Inc., New
  York, 1958),} We suppose that the distribution function \(W_{N}\) from
which \(H_{G}\) and \(W\left( \epsilon \right)\) are computed is either
a canonical distribution or a time- developed version of one resulting
from some dynamical perturbation; and that the system is such that the
canonical ensemble predicts relative fluctuations in energy which tend
to zero as \(N^{- 1/2}\) in the ``thermodynamic limit'' as
\(N \rightarrow \infty\) at constant density. The Gibbs \(H\) per
particle, \(H_{G}/N\), then approaches a definite limit, and

\begin{equation}
\lim_{N \rightarrow \infty}\left\{ \left\lbrack H_{G} + \log{W(\epsilon)} \right\rbrack/N \right\} = 0
\label{eqn-eighteen}
\end{equation}

provided \(\epsilon\) is not zero or unity. The principal feature of
this theorem, at first sight astonishing, is that the result is
independent of \(\epsilon\). Changing \(\epsilon\) does, of course,
change \(W\left( \epsilon \right)\); and generally by an enormous
factor. But the change in \(\log{W(\epsilon)}\) grows less rapidly than
\(N\) and in the limit it makes no difference.

The intuitive meaning of this theorem is that the Gibbs \(H\)
\emph{does} measure the logarithm of phase volume of reasonably probable
microstates and, remarkably, for a large system the amount per particle,
\(\log{W(\epsilon)/N}\), becomes independent of just what we mean by
``reasonably probable.'' We are thus able to retain Boltzmann's original
formula, \(S = k\log W\), which is seen to be precisely related to the
Gibbs \(H\)\emph{\textsc{,}} not the Boltzmann one.

With this interpretation of entropy, let us reconsider the above
experiment. At time \(t = 0\), we measure a number of macroscopic
parameters \(\left\{ X_{1}\left( 0 \right),\ \ldots,X_{n}(0) \right\}\)
adequate to define the thermodynamic state. The corresponding canonical
distribution determines a high-probability region \(R_{0}\)\emph{,} of
phase volume \(W_{0}\). The aforementioned variational property of the
canonical ensemble now implies that, of all ensembles agreeing with this
initial data in the sense of (\ref{eqn-sixteen}), the canonical one defines the
\emph{largest} high-probability region. The phase volume \(W_{0}\)
therefore describes the full range of possible initial microstates; and
not some arbitrary subset of them; this is the basic justification for
using the canonical distribution to describe partial information.

On the ``subjective'' side, we can therefore say that \(W_{0}\) measures
our \emph{degree of ignorance} as to the true unknown microstate, when
the only information we have consists of the macroscopic thermodynamic
parameters; a remark first made by Boltzmann.

But, and perhaps more pertinent, we can also say on the ``objective''
side, that \(W_{0}\) measures the \emph{degree of control of the
experimenter over the microstate}, when the only parameters he can
manipulate are the usual macroscopic ones. On successive repetitions of
the experiment, the initial microstate will surely not be repeated; it
will vary at random over the high-probability region \(R_{0}\)\emph{.}

When we carry out an adiabatic change of state, the region \(R_{0}\) is
transformed, by the equations of motion, into a new region
\(R_{t}\). From either the constancy of
\(H_{G}\), or directly from Liouville's theorem, the
phase volume remains unchanged; \(W_{t} = W_{0}\). Each possible
initial microstate in \(R_{0}\) uniquely determines a possible final one
in \(R_{t}\), and on successive repetitions of the experiment, the final
state varies over \(R_{t}\) at random.

At the end of this experiment, under the new equilibrium conditions, we
note the new values
\(\left\{ X_{1}\left( t \right),\ldots,X_{n}\left( t \right) \right\}\)
of the thermodynamic quantities. Now consider the region \(R^{\prime}\)
consisting of all microstates that are compatible with these new
\(X_{i}\left( t \right)\), whether or not they could have resulted from
the experiment just described; i.e., whether or not they also lie in
\(R_{t}\). By (\ref{eqn-seventeen}) and (\ref{eqn-eighteen}), the final experimental entropy is
\(S_{c}^{\prime} = k\log{W^{\prime}}\), where \(W^{\prime}\) is the phase volume of \(R^{\prime}\);
the experimental entropy is a measure of all conceivable ways in which
the final macrostate can be realized, and not merely of all ways in
which it could be produced in one particular experiment.

Now it is obvious that, if the observed change of state
\(X_{i}\left( 0 \right) \rightarrow X_{i}\left( t \right)\) is to be
experimentally reproducible, the region \(R_{t}\) resulting from the
experiment must be totally contained in \(R^{\prime}\). But this is possible
only if the phase volumes satisfy \(W_{t} \leq W^{\prime}\), which is again the
second law!

\emph{At this point, we finally see the real reason for the second law;
since phase volume is conserved in the dynamical evolution,} it is a
fundamental requirement on any reproducible process that the phase
volume \(W^{\prime}\) compatible with the final state cannot be less than the
phase volume \(W_{0}\) which describes our ability to reproduce the
initial state.

But this argument has given us more than the second law; in the past the
notion ``experimental entropy'' has been defined, in conventional
thermodynamics, only for \emph{equilibrium} states. It is suddenly clear
that the second law is only a very special case of a general restriction
on the direction of any reproducible process, whether or not the initial
and final states are describable in the language of thermodynamics; the
expression \(S = k\log W\) gives a generalized definition of entropy
applicable to arbitrary nonequilibrium states, which still has the
property that it can only increase in a reproducible experiment. This
can be shown directly from Liouville's theorem, without any
consideration of canonical distributions or the asymptotic equipartition
theorem.

Finally, it is clear that this extension of the second law can be
subjected to experimental tests.

Returning to the case of equilibrium thermodynamics, these
considerations (which are easily extended{[}\^{}2{]} to quantum
statistics) lead us to state the conventional second law in the form:
\emph{The experimental entropy cannot decrease in a reproducible
adiabatic process that starts from a state of complete thermal
equilibrium}.

The necessity of the last proviso is clear from a logical standpoint in
our derivation of the second law in Sec. \ref{sec-the-second-law}; for if the preparation of
the system just before \(t = 0\) imposes any constraints other than
those implied by the canonical distribution, the manifold of possible
initial states will be reduced below \(W_{0}\), and we shall not have an
equality in Eq. (\ref{eqn-seventeen}) initially. This necessity is also shown strikingly
from an experimental standpoint in the phenomenon of spin
echos,\footnote{E. L. Hahn, Phys. Rev. 80, 580 (1950).},\footnote{A. L
  Bloom, Phys. Rev. 98, 1104 (1955).} which is a gross violation of any
statement of the second law that fails to specify anything about the
past history of the system. This proviso has not been particularly
emphasized before, but it has always been obvious that some such
condition would be needed before we had a really air-tight statement of
the second law, which could not be violated by a clever experimenter.
The future behavior of the system is uniquely determined, according to
the laws of mechanics, only when one has specified perhaps \(10^{24}\)
microscopic coordinates and momenta; it could not possibly be determined
merely by the values of the three or four quantities measured in typical
thermodynamic experiments.

Specifying ``complete thermal equilibrium'' is still not as precise a
statement as we might wish. Experimentally, the only criterion as to
whether it is satisfied seems to be that the system is ``aged,'' i.e.,
that it is quiescent, the macroscopic quantities \(X_{i}\) unchanging,
for a sufficiently long time; and only experience can tell the
experimenter how long is ``sufficiently long.''

Theoretically, we can understand this requirement as meaning that, for
purposes of prediction, lack of knowledge of the present microstate can
be, in part, compensated by knowledge of the past history of the
macroscopic state. As we observe the system to be quiescent for a longer
and longer time, we become more and more confident that it is not in an
atypical microstate that will lead to ``abnormal'' behavior in the
future. In Hahn's experiment\textsuperscript{8} the spin system, having
no observable net magnetization at time \(t = 0\), is nevertheless able
to develop, spontaneously and almost magically, a large and reproducible
magnetization at a later time only because it ``remembers'' some very
atypical things that were done to it before \(t = 0\).

In this observation lies the clue that shows how to extend the
mathematical methods of Gibbs to a general formalism for predicting
irreversible phenomena; we must learn how to construct ensembles which
describe not only the present values of macroscopic quantities, but also
whatever information we have about their past behavior. The details of
this generalization will be given elsewhere.

\section{The ``Anthropomorphic'' Nature of Entropy}\label{sec-the-anthropomorphic-nature-of-entropy}

After the above insistence that any demonstration of the second law must
involve the entropy as measured experimentally, it may come as a shock
to realize that, nevertheless, thermodynamics knows of no such notion as
the ``entropy of a physical system.'' Thermodynamics does have the
concept of the entropy of a \emph{thermodynamic} system; but a given
physical system corresponds to many different thermodynamic systems.

Consider, for example, a crystal of Rochelle salt. For one set of
experiments on it, we work with temperature, pressure, and volume. The
entropy can be expressed as some function \(S_{0}(T,P)\). For another
set of experiments on the same crystal, we work with temperature, the
component \(e_{\text{xy}}\) of the strain tensor, and the component
\(P_{z}\) of electric polarization; the entropy as found in these
experiments is a function \(S_{0}(T,\ e_{\text{xy}},\ P_{z})\). It is
clearly meaningless to ask, ``What is the entropy of the crystal?''
unless we first specify the set of parameters which define its
thermodynamic state.

One might reply that in each of the experiments cited, we have used only
part of the degrees of freedom of the system, and there is a ``true''
entropy which is a function of all these parameters simultaneously.
However, we can always introduce as many new degrees of freedom as we
please. For example, we might expand each element of the strain tensor
in a complete orthogonal set of functions \(\varphi(x,y,x)\)
%
\[
e_{\text{ij}}\left( x,y,z \right) = \sum_{k}^{}{a_{\text{ijk}}\varphi_{k}(x,y,z)}
\]
%
and by a sufficiently complicated system of levels, we could vary each
of the first 1000 expansion coefficients \(a_{\text{ijk}}\)
independently. Our crystal is now a thermodynamic system of over 1000
degrees of freedom; but we still believe that the laws of thermodynamics
would hold. So, the entropy must be a function of over 1000 independent
variables. There is no end to this search for the ultimate ``true''
entropy until we have reached the point where we control the location of
each atom independently. But just at that point the notion of entropy
collapses, and we are no longer talking thermodynamics!

From this we see that entropy is an anthropomorphic concept, not only in
the well-known statistical sense that it measures the extent of human
ignorance as to the microstate. \emph{Even at the purely
phenomenological level}, \emph{entropy is an anthropomorphic concept.}
For it is a property, not of the physical system, but of the particular
experiments you or 1 choose to perform on it.

This points up still another qualification on the statement of the
second law without which it is, strictly speaking, no law at all. If we
work with a \emph{thermodynamic} system of \emph{n} degrees of freedom,
the experimental entropy is a function \(S_{0}(X_{1}\ldots X_{n})\) of
\(n\) independent variables. But the \emph{physical} system has any
number of additional degrees of freedom \(X_{n + 1},\ X_{n + 2},\) etc.
We have to understand that these additional degrees of freedom are not
to be tampered with during the experiments on the \(n\) degrees of
interest; otherwise one could easily produce apparent violations of the
second law.

For example, the engineers have their ``steam tables,'' which give
measured values of the entropy of superheated steam at various
temperatures and pressures. But the H\textsubscript{2}0 molecule has a
large electric dipole moment; and so the entropy of steam depends
appreciably on the electric field strength present. It must always be
understood implicitly (because it is never stated explicitly) that this
extra thermodynamic degree of freedom was not tampered with during the
experiments on which the steam tables are based; which means, in this
case, that the electric field was not inadvertently varied from one
measurement to the next.

Recognition that the ``entropy of a physical system'' is not meaningful
without further qualifications is important in clarifying many questions
concerning irreversibility and the second law. For example, I have been
asked several times whether, in my opinion, a biological system, say a
cat, which converts inanimate food into a highly organized structure and
behavior, represents a violation of the second law. The answer I always
give is that, until we specify the set of parameters which define the
\emph{thermodynamic state} of the cat, no definite question has been
asked!

It seems apparent, in view of complications which we have encountered in
the attempt to give a complete statement of the second law, that much
more work needs to be done in this field. Glib, unqualified statements
to the effect that ``entropy measures randomness'' are in my opinion
totally meaningless, and present a serious barrier to any real
understanding of these problems. A full resolution of all the questions
that can be raised requires a much more careful analysis than any that
has been attempted thus far. Perhaps the most difficult problem of all
is to learn how to state clearly \emph{what is the specific question we
are trying to answer}? However, I believe that in the above arguments we
have been able to report some progress in this direction.

\section{Acknowledgments}\label{sec-acknowledgments}

I have profited from discussions of these problems, over many years,
with Professor E. P. Wigner, from whom I first heard the remark,
``Entropy is an anthropomorphic concept.'' It is a pleasure to thank
Professor Wm. C. Band for reading a preliminary draft of this article,
and suggesting an important clarification of the argument.

\end{document}
